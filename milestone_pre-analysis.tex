%%%%%%%% ICML 2025 EXAMPLE LATEX SUBMISSION FILE %%%%%%%%%%%%%%%%%

\documentclass{article}

% Recommended, but optional, packages for figures and better typesetting:
\usepackage{microtype}
\usepackage{graphicx}
\usepackage{subfigure}
\usepackage{booktabs} % for professional tables

% hyperref makes hyperlinks in the resulting PDF.
% If your build breaks (sometimes temporarily if a hyperlink spans a page)
% please comment out the following usepackage line and replace
% \usepackage{icml2025} with \usepackage[nohyperref]{icml2025} above.
\usepackage{hyperref}


% Attempt to make hyperref and algorithmic work together better:
\newcommand{\theHalgorithm}{\arabic{algorithm}}

% Use the following line for the initial blind version submitted for review:
% \usepackage{icml2025}

% If accepted, instead use the following line for the camera-ready submission:
\usepackage[accepted]{icml2025}

% For theorems and such
\usepackage{amsmath}
\usepackage{amssymb}
\usepackage{mathtools}
\usepackage{amsthm}

% if you use cleveref..
\usepackage[capitalize,noabbrev]{cleveref}

%%%%%%%%%%%%%%%%%%%%%%%%%%%%%%%%
% THEOREMS
%%%%%%%%%%%%%%%%%%%%%%%%%%%%%%%%
\theoremstyle{plain}
\newtheorem{theorem}{Theorem}[section]
\newtheorem{proposition}[theorem]{Proposition}
\newtheorem{lemma}[theorem]{Lemma}
\newtheorem{corollary}[theorem]{Corollary}
\theoremstyle{definition}
\newtheorem{definition}[theorem]{Definition}
\newtheorem{assumption}[theorem]{Assumption}
\theoremstyle{remark}
\newtheorem{remark}[theorem]{Remark}

% Todonotes is useful during development; simply uncomment the next line
%    and comment out the line below the next line to turn off comments
%\usepackage[disable,textsize=tiny]{todonotes}
\usepackage[textsize=tiny]{todonotes}


% The \icmltitle you define below is probably too long as a header.
% Therefore, a short form for the running title is supplied here:
\icmltitlerunning{Milestone 2: Pre-Analysis Plan}

\begin{document}

\twocolumn[
\icmltitle{Milestone 2: \\
            Pre-Analysis Plan}

% It is OKAY to include author information, even for blind
% submissions: the style file will automatically remove it for you
% unless you've provided the [accepted] option to the icml2025
% package.

% List of affiliations: The first argument should be a (short)
% identifier you will use later to specify author affiliations
% Academic affiliations should list Department, University, City, Region, Country
% Industry affiliations should list Company, City, Region, Country

% You can specify symbols, otherwise they are numbered in order.
% Ideally, you should not use this facility. Affiliations will be numbered
% in order of appearance and this is the preferred way.
\icmlsetsymbol{equal}{*}

\begin{icmlauthorlist}
\icmlauthor{Shawn Malik}{equal,yyy}
\icmlauthor{Huyen Huynh}{equal,yyy}
\icmlauthor{Wenwan Xu}{equal,yyy}
\end{icmlauthorlist}

\icmlaffiliation{yyy}{University of Virginia, Charlottesville, Virginia, USA}

% \icmlcorrespondingauthor{Firstname1 Lastname1}{first1.last1@xxx.edu}
% \icmlcorrespondingauthor{Firstname2 Lastname2}{first2.last2@www.uk}

% You may provide any keywords that you
% find helpful for describing your paper; these are used to populate
% the "keywords" metadata in the PDF but will not be shown in the document
\icmlkeywords{Machine Learning, ICML}

\vskip 0.3in
]

% this must go after the closing bracket ] following \twocolumn[ ...

% This command actually creates the footnote in the first column
% listing the affiliations and the copyright notice.
% The command takes one argument, which is text to display at the start of the footnote.
% The \icmlEqualContribution command is standard text for equal contribution.
% Remove it (just {}) if you do not need this facility.

%\printAffiliationsAndNotice{}  % leave blank if no need to mention equal contribution
\printAffiliationsAndNotice{\icmlEqualContribution} % otherwise use the standard text.

%\begin{abstract}

%\end{abstract}

\section{Data}
\label{sec:data}

\subsection*{Sources and Scope}
All data, reports, and code will be at the link below:

\url{https://github.com/shawnmalik1/DS3001_F25_Project}

This project seeks to answer the question: Which performance metrics best predict NBA player salaries?

This project combines two season-level player files:
\begin{itemize}
    \item \textbf{Player Performance} (\texttt{player\_stats.csv}): per-player statistics of NBA players for season 2024--25 from Basketball-Reference, including counting metrics (e.g., \texttt{G}, \texttt{MP}), efficiency and usage rates (e.g., \texttt{TS\%}, \texttt{3PAr}, \texttt{FTr}, \texttt{USG\%}), play-type rate stats (e.g., \texttt{AST\%}, \texttt{TRB\%}, \texttt{STL\%}, \texttt{BLK\%}, \texttt{TOV\%}), and all-in metrics (e.g., \texttt{PER}, \texttt{WS}, \texttt{WS/48}, \texttt{OBPM}, \texttt{DBPM}, \texttt{BPM}, \texttt{VORP}), plus demographics/role (\texttt{Age}, \texttt{Pos}, \texttt{Team}).
    
    \item \textbf{Player Salary} (\texttt{player\_salaries.csv}): per-player base-salary information from Basketball-Reference by future season (e.g., 2025--26, 2026--27, \ldots) and a total guaranteed money.
\end{itemize}

Both files are sourced from Basketball-Reference, which keeps naming and IDs consistent across statistics and contracts.
The modeling target is \emph{next season's} base salary (2025--26), and predictors are taken from the immediately preceding season's on-court performance. This aligns the temporal sequence of ``performance $\rightarrow$ pay.''

\subsection*{Unit of Analysis and Coverage}
Basketball-Reference reports one row per team stint and, for traded players, sometimes an aggregated multi-team line (e.g. ``2TM / 3TM''). To keep things simple and consistent, we define one row per player-season by selecting the \textbf{single team stint with the most minutes} (\texttt{MP}) for each player. This yields a representative stat line while avoiding double-counting and extra aggregation steps if a player played for a team and then got traded to another team in the same season. We will use the team that had the highest stint for that player.

After this filtering, we merge the stats and salary files and retain only players with both a prior-season stat line and a 2025--26 base salary. Not all players will meet both conditions, so some players will drop because they lack a next-season salary (unsigned, two-way/10-day conversions) or did not log meaningful minutes in the prior season.

\subsection*{Identifiers and Merge Strategy}
When available, we use the Basketball-Reference player ID (e.g., \texttt{lillada01}) to create a common key \texttt{bbref\_id} and perform an \textbf{inner join}. If an ID is missing for a small number of rows, we fall back to a name-based merge after light name cleaning (standardizing suffixes such as ``Jr.''/``III'').  The team variable that appears in both datasets (\texttt{Team} in \texttt{player\_stats.csv} and \texttt{Tm} in \texttt{player\_salaries.csv}) also helps us to double-check the merging process after we rename them to be the same. 
Using IDs where possible minimizes mismatches from suffixes, abbreviations, and nicknames.

\subsection*{Key Variables}
\textbf{Base salary (2025--26) ($Y$)}: parsed from currency strings into numeric values (USD). The distribution is strongly right-skewed: many players have salaries around the league minimum, and a long tail up to super max-level salaries ($\sim\$60$M).

\paragraph{Candidate predictors (\(X\))}
\begin{itemize}
    \item \textbf{Availability \& playing time:} \texttt{G} (Games), \texttt{MP} (Minutes Played). Minutes both signal value and stabilize stats.
    \item \textbf{Efficiency \& usage:} \texttt{TS\%} (True Shooting Percentage), \texttt{3PAr} (3-Point Attempt Rate), \texttt{FTr} (Free Throw Attempt Rate), \texttt{USG\%} (Usage Percentage).
    \item \textbf{Role/rate indicators:} \texttt{AST\%} (Assist Percentage), \texttt{TRB\%} (Total Rebound Percentage, which could be specified through \texttt{ORB\%}, Offensive Rebound Percentage, and \texttt{DRB\%}, Defensive Rebound Percentage), \texttt{STL\%} (Steal Percentage), \texttt{BLK\%} (Block Percentage), \texttt{TOV\%} (Turnover Percentage).
    \item \textbf{Overall metrics:} \texttt{PER} (Player Efficiency Rating), \texttt{WS} (Win Shares), \texttt{WS/48} (Win Shares Per 48 Minutes), \texttt{OBPM} (Offensive Box Plus/Minus), \texttt{DBPM} (Defensive Box Plus/Minus), \texttt{BPM} (Box Plus/Minus), \texttt{VORP} (Value over Replacement Player).
    \item \textbf{Demographics/role:} \texttt{Age}, \texttt{Pos} (Position). Team indicators are available if market/team effects are modeled.
\end{itemize}

\subsection*{Reading, Cleaning, and Preparation}
\begin{enumerate}
    \item  \textbf{Read data from a single source.} Sourcing from Basketball-Reference~\cite{bbref_contracts,bbref_advanced_stats_2025}, which is the single database for both of our data sets keeps identifiers consistent. Reading in as \texttt{csv} files allows for straightforward integration and ensures compatibility with subsequent data processing in Python. 
    \item \textbf{Consolidate multi-team seasons.} Retain the consolidated ``xTM'' line; otherwise, keep the highest-\texttt{MP} stint. This yields exactly one prior-season stat line per player and prevents double-counting.
    \item \textbf{Parse money fields.} Strip ``\$'' and commas, coerce to numeric. Use 2025--26 base salary as the target; guaranteed totals span multiple years/options and are not modeled directly.
    \item \textbf{Handle missing data.} Drop rows missing the target salary or essential stats (e.g., \texttt{MP}, \texttt{Age}). Leave rarely used or mostly-empty fields (e.g., awards) out of the model.
    \item \textbf{Basic feature set.} Use straightforward predictors: \texttt{MP}, \texttt{G}, \texttt{TS\%}, \texttt{USG\%}, \texttt{AST\%}, \texttt{TRB\%}, \texttt{STL\%}, \texttt{BLK\%}, \texttt{TOV\%}, and one or two overall metrics (e.g., \texttt{BPM}, \texttt{WS}). Include \texttt{Age} and \texttt{Pos}. 

\end{enumerate}

\section{Methods and Results}
\label{sec:Methods and Results}

\subsection*{Method Overview}
Our goal is to predict each player’s base salary for the 2025--26 NBA season using performance metrics from the 2024--25 season. 
We treat this as a supervised regression problem where the target variable is salary (numeric, in USD) and the predictors are player-level statistics (e.g., minutes, efficiency, usage, and overall metrics).

A basic exploratory data analysis (EDA) will be performed to examine relationships between the target and predictor variables. Visualizations such as histograms, scatter plots, and box plots will be used to assess variable distributions and potential associations. They could also provide some hints on potential transformation, and if there are any outliers or influential observations in our data. A correlation analysis will also be conducted to identify linear relationships and to consider removing predictors with weak correlations to the target variable, while remaining cautious of potential non-linear effects. An interaction term assessment will also be performed, as many basketball statistics are highly correlated (for example, \texttt{WS} and \texttt{BPM}), and exploring these interactions can help reveal combined effects among predictors before applying regularization methods such as Ridge and Lasso regression to address multicollinearity and reduce overfitting. 

We plan to start with simple, interpretable models and gradually test more flexible ones. This helps us understand the data patterns and avoid overfitting.

This approach allows us to build understanding in stages: starting from interpretable linear models and extending to more flexible ensemble methods. The analysis will not only assess predictive accuracy but also interpret how various player performance indicators contribute to contract value. For example, we expect that players with strong efficiency metrics and playmaking roles may command higher salaries even if their raw counting stats (e.g., points or rebounds) are moderate. The results of early models will guide which variables or transformations to emphasize in subsequent iterations.

\subsection*{Models}

1. \textbf{Linear Regression.}  
A basic linear model will serve as our starting point. It shows how each variable (like minutes or win shares) is related to salary, holding others constant.

2. \textbf{Ridge and Lasso Regression.}  
Because many basketball stats are correlated (for example, \texttt{WS} and \texttt{BPM}), we will use Ridge and Lasso regression to handle multicollinearity and reduce overfitting. Lasso may also help identify which variables matter most by shrinking weaker predictors toward zero.

3. \textbf{Random Forest.}  
We will test a Random Forest model to see if nonlinear patterns improve predictions. This model can capture interactions and give a ranking of variable importance.

\subsection*{Model Training Procedure}

We will split the data into a training set (80\%) and a test set (20\%).  
All numeric predictors will be standardized so that variables with large scales (like minutes played) do not dominate smaller-scale metrics (like percentages).

Models will be trained using the \texttt{scikit-learn} package in Python. 
For Ridge and Lasso, we will tune the regularization parameter using 5-fold cross-validation on the training set.

Categorical variables such as player position will be one-hot encoded before model fitting. In addition, continuous variables with skewed distributions (such as salary and minutes) may be log-transformed to stabilize variance and improve model fit. The model pipeline will be implemented using the \texttt{scikit-learn} framework, allowing for standardized preprocessing, fitting, and evaluation steps across all model types.


\subsection*{Model Validation Plan}

We will evaluate model performance on the test set using:
\begin{itemize}
    \item Root Mean Squared Error (RMSE)
    \item Mean Absolute Error (MAE)
    \item $R^2$ (coefficient of determination)
\end{itemize}

We will also compare predicted vs. actual salaries in scatter plots to see whether the model systematically over- or under-predicts certain players (for example, stars vs. bench players).

As for our \textbf{linear model}, a model assumption assessment should also be implemented to check if the regression assumptions are met. This would at least include a residual plot for the linear relationship and constant variance assumptions, a QQ plot for normality assumption, and a potential ACF plot to check independence of the data.  

\subsection*{Next Steps}

Once models are trained and validated, we will compare them and interpret which performance metrics best explain or predict player salary. 

We expect that metrics capturing playing time (\texttt{MP}), overall impact (\texttt{WS}, \texttt{BPM}, \texttt{VORP}), and efficiency (\texttt{TS\%}) will emerge as the strongest predictors.









\bibliography{milestone_data}
\bibliographystyle{icml2025}


%%%%%%%%%%%%%%%%%%%%%%%%%%%%%%%%%%%%%%%%%%%%%%%%%%%%%%%%%%%%%%%%%%%%%%%%%%%%%%%
%%%%%%%%%%%%%%%%%%%%%%%%%%%%%%%%%%%%%%%%%%%%%%%%%%%%%%%%%%%%%%%%%%%%%%%%%%%%%%%
% APPENDIX
%%%%%%%%%%%%%%%%%%%%%%%%%%%%%%%%%%%%%%%%%%%%%%%%%%%%%%%%%%%%%%%%%%%%%%%%%%%%%%%
%%%%%%%%%%%%%%%%%%%%%%%%%%%%%%%%%%%%%%%%%%%%%%%%%%%%%%%%%%%%%%%%%%%%%%%%%%%%%%%

%%%%%%%%%%%%%%%%%%%%%%%%%%%%%%%%%%%%%%%%%%%%%%%%%%%%%%%%%%%%%%%%%%%%%%%%%%%%%%%
%%%%%%%%%%%%%%%%%%%%%%%%%%%%%%%%%%%%%%%%%%%%%%%%%%%%%%%%%%%%%%%%%%%%%%%%%%%%%%%


\end{document}


% This document was modified from the file originally made available by
% Pat Langley and Andrea Danyluk for ICML-2K. This version was created
% by Iain Murray in 2018, and modified by Alexandre Bouchard in
% 2019 and 2021 and by Csaba Szepesvari, Gang Niu and Sivan Sabato in 2022.
% Modified again in 2023 and 2024 by Sivan Sabato and Jonathan Scarlett.
% Previous contributors include Dan Roy, Lise Getoor and Tobias
% Scheffer, which was slightly modified from the 2010 version by
% Thorsten Joachims & Johannes Fuernkranz, slightly modified from the
% 2009 version by Kiri Wagstaff and Sam Roweis's 2008 version, which is
% slightly modified from Prasad Tadepalli's 2007 version which is a
% lightly changed version of the previous year's version by Andrew
% Moore, which was in turn edited from those of Kristian Kersting and
% Codrina Lauth. Alex Smola contributed to the algorithmic style files.
