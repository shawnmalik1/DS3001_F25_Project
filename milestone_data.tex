%%%%%%%% ICML 2025 EXAMPLE LATEX SUBMISSION FILE %%%%%%%%%%%%%%%%%

\documentclass{article}

% Recommended, but optional, packages for figures and better typesetting:
\usepackage{microtype}
\usepackage{graphicx}
\usepackage{subfigure}
\usepackage{booktabs} % for professional tables

% hyperref makes hyperlinks in the resulting PDF.
% If your build breaks (sometimes temporarily if a hyperlink spans a page)
% please comment out the following usepackage line and replace
% \usepackage{icml2025} with \usepackage[nohyperref]{icml2025} above.
\usepackage{hyperref}


% Attempt to make hyperref and algorithmic work together better:
\newcommand{\theHalgorithm}{\arabic{algorithm}}

% Use the following line for the initial blind version submitted for review:
% \usepackage{icml2025}

% If accepted, instead use the following line for the camera-ready submission:
\usepackage[accepted]{icml2025}

% For theorems and such
\usepackage{amsmath}
\usepackage{amssymb}
\usepackage{mathtools}
\usepackage{amsthm}

% if you use cleveref..
\usepackage[capitalize,noabbrev]{cleveref}

%%%%%%%%%%%%%%%%%%%%%%%%%%%%%%%%
% THEOREMS
%%%%%%%%%%%%%%%%%%%%%%%%%%%%%%%%
\theoremstyle{plain}
\newtheorem{theorem}{Theorem}[section]
\newtheorem{proposition}[theorem]{Proposition}
\newtheorem{lemma}[theorem]{Lemma}
\newtheorem{corollary}[theorem]{Corollary}
\theoremstyle{definition}
\newtheorem{definition}[theorem]{Definition}
\newtheorem{assumption}[theorem]{Assumption}
\theoremstyle{remark}
\newtheorem{remark}[theorem]{Remark}

% Todonotes is useful during development; simply uncomment the next line
%    and comment out the line below the next line to turn off comments
%\usepackage[disable,textsize=tiny]{todonotes}
\usepackage[textsize=tiny]{todonotes}


% The \icmltitle you define below is probably too long as a header.
% Therefore, a short form for the running title is supplied here:
\icmltitlerunning{Milestone 1: Data Description}

\begin{document}

\twocolumn[
\icmltitle{Milestone 1: \\
            Data Description}

% It is OKAY to include author information, even for blind
% submissions: the style file will automatically remove it for you
% unless you've provided the [accepted] option to the icml2025
% package.

% List of affiliations: The first argument should be a (short)
% identifier you will use later to specify author affiliations
% Academic affiliations should list Department, University, City, Region, Country
% Industry affiliations should list Company, City, Region, Country

% You can specify symbols, otherwise they are numbered in order.
% Ideally, you should not use this facility. Affiliations will be numbered
% in order of appearance and this is the preferred way.
\icmlsetsymbol{equal}{*}

\begin{icmlauthorlist}
\icmlauthor{Shawn Malik}{equal,yyy}
\icmlauthor{Wenwan Xu}{equal,yyy}
\icmlauthor{Huyen Huynh}{equal,yyy}
\end{icmlauthorlist}

\icmlaffiliation{yyy}{University of Virginia, Charlottesville, Virginia, USA}

% \icmlcorrespondingauthor{Firstname1 Lastname1}{first1.last1@xxx.edu}
% \icmlcorrespondingauthor{Firstname2 Lastname2}{first2.last2@www.uk}

% You may provide any keywords that you
% find helpful for describing your paper; these are used to populate
% the "keywords" metadata in the PDF but will not be shown in the document
\icmlkeywords{Machine Learning, ICML}

\vskip 0.3in
]

% this must go after the closing bracket ] following \twocolumn[ ...

% This command actually creates the footnote in the first column
% listing the affiliations and the copyright notice.
% The command takes one argument, which is text to display at the start of the footnote.
% The \icmlEqualContribution command is standard text for equal contribution.
% Remove it (just {}) if you do not need this facility.

%\printAffiliationsAndNotice{}  % leave blank if no need to mention equal contribution
\printAffiliationsAndNotice{\icmlEqualContribution} % otherwise use the standard text.

%\begin{abstract}

%\end{abstract}

\section{Data}
\label{sec:data}

\subsection*{Sources and Scope}
All data, reports, and code will be at the link below:

\url{https://github.com/shawnmalik1/DS3001_F25_Project}

This project seeks to answer the question: Which performance metrics best predict NBA player salaries?

This project combines two season-level player files:
\begin{itemize}
    \item \textbf{Player Performance} (\texttt{player\_stats.csv}): per-player statistics of NBA players for season 2024--25 from Basketball-Reference, including counting metrics (e.g., \texttt{G}, \texttt{MP}), efficiency and usage rates (e.g., \texttt{TS\%}, \texttt{3PAr}, \texttt{FTr}, \texttt{USG\%}), play-type rate stats (e.g., \texttt{AST\%}, \texttt{TRB\%}, \texttt{STL\%}, \texttt{BLK\%}, \texttt{TOV\%}), and all-in metrics (e.g., \texttt{PER}, \texttt{WS}, \texttt{WS/48}, \texttt{OBPM}, \texttt{DBPM}, \texttt{BPM}, \texttt{VORP}), plus demographics/role (\texttt{Age}, \texttt{Pos}, \texttt{Team}).
    
    \item \textbf{Player Salary} (\texttt{player\_salaries.csv}): per-player base-salary information from Basketball-Reference by future season (e.g., 2025--26, 2026--27, \ldots) and a total guaranteed money.
\end{itemize}

Both files are sourced from Basketball-Reference, which keeps naming and IDs consistent across statistics and contracts.
The modeling target is \emph{next season's} base salary (2025--26), and predictors are taken from the immediately preceding season's on-court performance. This aligns the temporal sequence of ``performance $\rightarrow$ pay.''

\subsection*{Unit of Analysis and Coverage}
Basketball-Reference reports one row per team stint and, for traded players, sometimes an aggregated multi-team line (e.g. ``2TM / 3TM''). To keep things simple and consistent, we define one row per player-season by selecting the \textbf{single team stint with the most minutes} (\texttt{MP}) for each player. This yields a representative stat line while avoiding double-counting and extra aggregation steps if a player played for a team and then got traded to another team in the same season. We will use the team that had the highest stint for that player.

After this filtering, we merge the stats and salary files and retain only players with both a prior-season stat line and a 2025--26 base salary. Not all players will meet both conditions, so some players will drop because they lack a next-season salary (unsigned, two-way/10-day conversions) or did not log meaningful minutes in the prior season.

\subsection*{Identifiers and Merge Strategy}
When available, we use the Basketball-Reference player ID (e.g., \texttt{lillada01}) to create a common key \texttt{bbref\_id} and perform an \textbf{inner join}. If an ID is missing for a small number of rows, we fall back to a name-based merge after light name cleaning (standardizing suffixes such as ``Jr.''/``III'').  The team variable that appears in both datasets (\texttt{Team} in \texttt{player\_stats.csv} and \texttt{Tm} in \texttt{player\_salaries.csv}) also helps us to double-check the merging process after we rename them to be the same. 
Using IDs where possible minimizes mismatches from suffixes, abbreviations, and nicknames.

\subsection*{Key Variables}
\textbf{Base salary (2025--26) ($y$)}: parsed from currency strings into numeric values (USD). The distribution is strongly right-skewed: many players have salaries around the league minimum, and a long tail up to super max-level salaries ($\sim\$60$M).

\paragraph{Candidate predictors (\(X\))}
\begin{itemize}
    \item \textbf{Availability \& playing time:} \texttt{G} (Games), \texttt{MP} (Minutes Played). Minutes both signal value and stabilize stats.
    \item \textbf{Efficiency \& usage:} \texttt{TS\%} (True Shooting Percentage), \texttt{3PAr} (3-Point Attempt Rate), \texttt{FTr} (Free Throw Attempt Rate), \texttt{USG\%} (Usage Percentage).
    \item \textbf{Role/rate indicators:} \texttt{AST\%} (Assist Percentage), \texttt{TRB\%} (Total Rebound Percentage, which could be specified through \texttt{ORB\%}, Offensive Rebound Percentage, and \texttt{DRB\%}, Defensive Rebound Percentage), \texttt{STL\%} (Steal Percentage), \texttt{BLK\%} (Block Percentage), \texttt{TOV\%} (Turnover Percentage).
    \item \textbf{Overall metrics:} \texttt{PER} (Player Efficiency Rating), \texttt{WS} (Win Shares), \texttt{WS/48} (Win Shares Per 48 Minutes), \texttt{OBPM} (Offensive Box Plus/Minus), \texttt{DBPM} (Defensive Box Plus/Minus), \texttt{BPM} (Box Plus/Minus), \texttt{VORP} (Value over Replacement Player).
    \item \textbf{Demographics/role:} \texttt{Age}, \texttt{Pos} (Position). Team indicators are available if market/team effects are modeled.
\end{itemize}

\subsection*{Reading, Cleaning, and Preparation}
\begin{enumerate}
    \item \textbf{Consolidate multi-team seasons.} Retain the consolidated ``xTM'' line; otherwise, keep the highest-\texttt{MP} stint. This yields exactly one prior-season stat line per player and prevents double-counting.
    \item \textbf{Parse money fields.} Strip ``\$'' and commas, coerce to numeric. Use 2025--26 base salary as the target; guaranteed totals span multiple years/options and are not modeled directly.
    \item \textbf{Handle missing data.} Drop rows missing the target salary or essential stats (e.g., \texttt{MP}, \texttt{Age}). Leave rarely used or mostly-empty fields (e.g., awards) out of the model.
    \item \textbf{Basic feature set.} Use straightforward predictors: \texttt{MP}, \texttt{G}, \texttt{TS\%}, \texttt{USG\%}, \texttt{AST\%}, \texttt{TRB\%}, \texttt{STL\%}, \texttt{BLK\%}, \texttt{TOV\%}, and one or two overall metrics (e.g., \texttt{BPM}, \texttt{WS}). Include \texttt{Age} and \texttt{Pos}.

All data were sourced from Basketball-Reference~\cite{bbref_contracts,bbref_advanced_stats_2025}.

\end{enumerate}


\bibliography{milestone_data}
\bibliographystyle{icml2025}


%%%%%%%%%%%%%%%%%%%%%%%%%%%%%%%%%%%%%%%%%%%%%%%%%%%%%%%%%%%%%%%%%%%%%%%%%%%%%%%
%%%%%%%%%%%%%%%%%%%%%%%%%%%%%%%%%%%%%%%%%%%%%%%%%%%%%%%%%%%%%%%%%%%%%%%%%%%%%%%
% APPENDIX
%%%%%%%%%%%%%%%%%%%%%%%%%%%%%%%%%%%%%%%%%%%%%%%%%%%%%%%%%%%%%%%%%%%%%%%%%%%%%%%
%%%%%%%%%%%%%%%%%%%%%%%%%%%%%%%%%%%%%%%%%%%%%%%%%%%%%%%%%%%%%%%%%%%%%%%%%%%%%%%

%%%%%%%%%%%%%%%%%%%%%%%%%%%%%%%%%%%%%%%%%%%%%%%%%%%%%%%%%%%%%%%%%%%%%%%%%%%%%%%
%%%%%%%%%%%%%%%%%%%%%%%%%%%%%%%%%%%%%%%%%%%%%%%%%%%%%%%%%%%%%%%%%%%%%%%%%%%%%%%


\end{document}


% This document was modified from the file originally made available by
% Pat Langley and Andrea Danyluk for ICML-2K. This version was created
% by Iain Murray in 2018, and modified by Alexandre Bouchard in
% 2019 and 2021 and by Csaba Szepesvari, Gang Niu and Sivan Sabato in 2022.
% Modified again in 2023 and 2024 by Sivan Sabato and Jonathan Scarlett.
% Previous contributors include Dan Roy, Lise Getoor and Tobias
% Scheffer, which was slightly modified from the 2010 version by
% Thorsten Joachims & Johannes Fuernkranz, slightly modified from the
% 2009 version by Kiri Wagstaff and Sam Roweis's 2008 version, which is
% slightly modified from Prasad Tadepalli's 2007 version which is a
% lightly changed version of the previous year's version by Andrew
% Moore, which was in turn edited from those of Kristian Kersting and
% Codrina Lauth. Alex Smola contributed to the algorithmic style files.
